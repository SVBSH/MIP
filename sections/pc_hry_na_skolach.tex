
\section{Počítačove hry na školách}\label{hry-na-skolach-hodiny-programovania}
Idea vzdelávania detí prostredníctvom počítačových hier v školách by mohla naraziť na 
rôzne problémy v dnešnom vzdelávacom programe. Učitelia na školách majú málo času, sú striktne 
nastavené smernice a hodnotenie učiteľov je podmienené výkonom. Otázka by mohla nastať pri chýbajúcich 
vývojároch hier, ktorí by museli pravidelne konzultovať prípadné konfigurácie náročnosti úrovní, na 
základe menej porozumených oblastiach učiva. Musela by sa vyriešiť rôznorodá gramotnosť detí v informačných 
technológií, ktorý by mal určite rôzne úrovne, už len vzhľadom na rôznu finančnú situáciu rodín.\\\\\\
\indent~Vo všeobecnosti je náročné udržiavať pozornosť detí, ale aktívnym hraním by to bolo jednoduchšie~\cite{6624228}. Hry 
deťom spôsobujú radosť, čo by zvyšovalo motiváciu k učeniu. Ďalší aspekt, ktorý by priniesla táto metóda vzdelávania 
je rozvinutie kreativity pri riešení rôznych hádaniek. Okrem hier individuálneho charakteru zamerania sa 
by mohol byť istý druh počítačových hier orientovaný na riešenie problémov v skupinách, čo by podnecovalo 
lepšiu kooperáciu.\cite{7795662} Tieto získané schopnosti by boli prospešné v ich kariérnej budúcnosti.\\
\indent~Ako príklad by som uviedol školu \textbf{\href{https://www.q2l.org/}{Q2L-Quest to Learn}} v New York City, kde už v roku 2009 sa 
zrodila myšlienka učenie hrou a odvtedy na danej škole prebieha špeciálna výuka od 6. do 12. ročníka. 
Vyučovací proces prostredníctvom rôznych hier sa ukazuje, ako účinná alternatíva klasického vyučovania.
Pozorovanie na danej škole ukazuje, že u ich študentov zaznamenali väčšiu mieru kooperácie. Ako výhodu v úlohách uvádzajú 
možnosť opravy, ak žiaci v hre zlyhajú, a tým sa zvyšuje motivácia skúšať znova prejsť danú výzvu, naučiť 
sa a byť nakoniec úspešný. \\
V knihe ``Moderní vyučování`` od spisovateľa Geoffrey Petty ~\cite{pettyEd}, sa tiež hovorí, že podľa názoru humanistických psychológov je učenie 
najľahšie, zmysluplné a najúčinnejšie vtedy, keď prebieha v atmosfére zbavenej akejkoľvek hrozby. Žiaci by nemali byť motivovaní strachom 
z neúspechu, ale túžbou uspieť, dozvedieť sa viac.“ Pedagógovia si pochvaľujú rýchlu spätnú väzbu od žiakov na prebranú látku. 
V roku 2015 v ELA skúškach mali ich žiaci nadpriemerné výsledky medzi študentmi z rôznych škôl v rámci mesta.\\
\indent~Vďaka myšlienke zavedenia hier do školského vzdelávacieho systému by pri dostatočných predpokladoch mohli nastať pozitívne 
zmeny vo výsledkoch študentov, v čom sa ukazuje nový potenciál aplikácie gemifikácie na školy.
