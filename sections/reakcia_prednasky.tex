\section{Reakcia na témy preberané na prednáškach} \label{reakcie_na_prednasky}

\paragraph{Spoločenské súvislosti.}
Budem reagovať predmet spomenutý na prednáške: Spoločenské súvislosti informatiky a informačných
a komunikačných technológií. Po získaní viacerých informácii o danom predmete som zistil že témy,
ktoré sa preberajú na danom predmete a taktiež aj písanie esejov nie sú blízke mojim záujmom. 


\paragraph{Historické súvislosti.}
V prednáške som sa dozvedel nové poznatky o výrobe čipov najviac ma konkrétne zaujali
videá o procese výroby čipov, konkrétne ~\textbf{\href{https://www.youtube.com/watch?v=u-dmL9ec26Q}{Intelového čipu}} a taktiež aj 
výroby~\textbf{\href{https://www.youtube.com/watch?v=vK-geBYygXo}{CPU}}.
Celkovo sa mi prednáška páčila aj keď ma mierne unavovala história no na druhej strane 
ma zaujali vyššie spomenuté videá a taktiež aj ukázaná realizácia logických hradiel.  

\paragraph{Technológia a ľudia.}
V prednáške bola predstavená metóda SCRUM ,ktorá je založená na iteratívnom
a inkrementálnom riadení projektov. Danú metódu som dovtedy nepoznal a prišla mi
inšpiratívna a zaujímavá hlavne iteratívne riadenie projektov. S metódou som doteraz nemal
skúsenosť ale rád by som ju videl aplikovanú aj v praxi.   

\paragraph{Udržateľnosť a etika.}
V prednáške som sa dozvedel poznatky o etike. Najviac ma zaujala časť o etike pri písaní 
softvéru a tvorbe dokumentácie. Celkovo hodnotím prednášku pozitívne.
