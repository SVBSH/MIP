\section{Úvod}
Počítačove hry sú v súčasnosti jedným z najväčších trhov na svete. Počet hodín strávených 
za hernými titulmi neustále rastie a taktiež rastie aj počet nových užívateľov hrajúcich 
herné tituly. Za posledné 3 roky sa situácia zmenila aj v školstve vplyvom pandémie COVID-19. Školy
boli zavraté a prezenčnú výučbu nahradilo distančné vzdelávanie. Učilia boli nútení k motivácii študentov využiť 
inovatívne riešenia k čomu výrazne pomohla gemifikácia výčbového procesu ~\cite{gemifikaciaPandemia, gemifikaciaNaSkole-1}.\\ 
Aj na základe vyššie spomenutých dát, ktoré budem neskoršie rozoberať v 
sekcii~\ref{data-o-hrani} by som sa chcel v tomto článku zamyslieť nad otázkou: 
„Je možne zvýšiť vzdelanie ludi hraním počítačových hier?“\\ 
V častiach~\ref{hry-na-skolach-hodiny-programovania} a~\ref{vytvorenie-vlastnej-hry} sa budem zaoberať 
tvorbou jednoduchých hier, ktorých cieľom bude naučenie sa nových spôsobov rozmýšľania nad problémami.
Záverečné poznámky opíšem v záverečnej časti~\ref{zaver}.