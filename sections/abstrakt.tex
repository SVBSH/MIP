\section{Abstrakt}\label{abstrakt}
\indent~Gemifikácia je efektívnym prístupupom k zvýšeniu motivovácie a následne aj výkonnosti
pri procese výučby. Prvky gemifikácie možno nájsť nielen pri počítačových hrách ale aj 
pri mnohých aplikáciach zameraných na vzdelávanie. Príklad gemifikácie môže byť tabuľka so skóre,
ktorá môže reprezentovať určitý hodnotiaci rebríček medzi užívateľmi s cieľom vytvoriť prostredie 
pre konkurenciu jednotlivých užívateľov. Toto prostredie má nielen zabezpečiť konkurenciu medzi používateľmi 
ale primárne ich motivovať k opätovnému opakovaniu nesplnených alebo nezvládnutých úloh
a k postupovaniu v rebríčku vzššie. Cieľom je vzdelať užívateľa aplikácie.
Na druhú stranu veľa štúdii zároveň zdôrazňuje, že pozitívny efekt gemifikácie
závisí hlavne od situácie alebo kontextu, v ktorom implementujeme prvky gemifikácie. 
Taktiež závisí aj na cieľovej skupine ľudí, ktorá by ňou mala byť ovplyvnená.
Tento článok bude rozoberať využitie gemifikácie v akademickej sfére.
